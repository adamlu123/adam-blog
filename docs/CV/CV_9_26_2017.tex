\documentclass[margin,line]{res}


% \documentclass[line,mm]{res}
\usepackage{hyperref}
\usepackage{url}
% \usepackage[compact]{titlesec}
\oddsidemargin -.5in
\evensidemargin -.5in
\textwidth=6.0in
\textheight = 700pt
\itemsep=0in
\parsep=0in
\topmargin=-0.3in
\topskip=0in
 
\newenvironment{list1}{
  \begin{list}{\ding{113}}{%
      \setlength{\itemsep}{0in}
      \setlength{\parsep}{0in} \setlength{\parskip}{0in}
      \setlength{\topsep}{0in} \setlength{\partopsep}{0in}
      \setlength{\leftmargin}{0.17in}}}{\end{list}}
\newenvironment{list2}{
  \begin{list}{$\bullet$}{%
      \setlength{\itemsep}{0in}
      \setlength{\parsep}{0in} \setlength{\parskip}{0in}
      \setlength{\topsep}{0in} \setlength{\partopsep}{0in}
      \setlength{\leftmargin}{0.2in}}}{\end{list}}


% \usepackage{anysize}
% \marginsize{0.4in}{0.4in}{0.4in}{0.4in}
% \usepackage[bottom=0.4in,top=0.5in]{geometry}
%\usepackage[bottom=3cm]{geometry}
\begin{document}

\name{\ Yadong  Lu } % \hfill For Applied Statistics M.S. Fall 2016. U-M ID:25172289 %\hfill {\em \today}可以加上今天的日期

\begin{resume}
\section{\sc Contact Information}

\vspace{.05in}
\begin{tabular}{p{3.5in}p{3in}}
2243 Donald Bren Hall & {Phone:}  (518)334-4626\\University of California, Irvine 
& {E-mail:}  yadongl1@uci.edu\\

\end{tabular}


\section{\sc Interests}
  Deep learning algorithms with applications to images and high energy physics, Causal inference.

\section{\sc Education}
{\bf University of California, Irvine}, Statistics, 2nd year PhD student \hfill   2016.9-now \\
GPA: 3.97/4.0, Advisor: Pierre Baldi \\
{\bf Sichuan University}, Mathematics,  B.S. \hfill  2012 - 2016\\
Honor Class, 4 year National Pilot Program Fellowship receipient. \\ 
% GPA: 3.72/4.0 (3rd in ranking)\\ %\vspace*{-.1in}
 %\hfill(GPA 3.70/4.0 (Grade in NUS included), Ranked 4/155) \\
{\bf National University of Singapore},  Applied Math \hfill   2014 - 2015 \\
Temasek Foundation Scholarship, one-year study abroad programme.
\vspace{2pt}
 %\hfill(GPA 4.43/5.0) \\%($A{+}$, 1 $A$ and 2 $A{-}s$)\\
%{\bf Yale University}, Department of statistics \hfill To be started from Sept 2016\\
%\hfill M.S. of Statistics.
%\hfill PhD Candidate.
\section{\sc Technical Skills}
{\bf Programming Languages}:  Fluent: Python, R; Intermidiate: Matlab, C, SAS\\
{\bf Deep Learning Related}: Tensorflow, Keras, Theano, Pandas, Linux
\section{\sc Test Scores}
{\bf GRE}: Quantitative Reasoning: 170/170, Verbal Reasoning: 157/170% {\bf TOEFL}: 106/12
\section{\sc Research and Intern Experiences}
%%%%%%%%%%%%%%%%%%%%%%
{\bf Sherpa: Hyperparameter Tuning Machine for Neural Networks}    \hfill{2017.9-now}
\begin{list2}
% \item Software development at Baldi's Lab. 
\item Sherpa aims to provide a fast and parallel automatic hyperparameter search framework for training various kinds of deep neural networks. 
\item Lead the development of bayesian optimization algorithms for Sherpa. 
\end{list2}


{\bf Antihydrogen Particle Detection by Deep Neural Network}    \hfill{2017.6-now}
\begin{list2}
\item A joint project with  ASACUSA at CERN.
\item Developed a package for analyzing over 3 million real annilation events generated from ASACUSA experiments. Wrote algorithms to efficiently separated the rising edges within the data.
\item Propsed resnet-like deep neural network as a novel technique to distinguish antimatter signals from backgrond noise. Our model outperforms the existing vertex construction algorithm.
\end{list2}

{\bf Deep Target Algorithms and Random Backpropagation}    \hfill{2016.9-12}
\begin{list2}
\item With Prof. Pierre Baldi at UCI.
\item Developed and implemented several deep target learning algorithms and Random Backpropagation (RBP) Rules in both Theano and Tenserflow. Tested the algorithms on several benchmark datasets using multiple GPUs.
\item Reseach on the proof of convergence properties of RBP algorithms. Constantly explore and evaluate new RBP rules.
\end{list2}


{\bf Quantative Trading Algorithms for Achieving Alpha}       \hfill{2016.3-6}
\begin{list2}
\item Internship at Huaxi Futures Co., Ltd. 
%\item With Prof. Chang Liu, at Sichuan University
%\item Stock portfolios were estabished in SSE by sorting them into quintiles according to influential factors such as EPS and money inflow rate.
\item Historical data within 3000 trading days was crawled from web using python to backtest trading strategies. Developed and implemented several algorithms to select profitable factors as trading strategies. 
\item Improved the automatic trading system by adding new strategies.
\end{list2}

\section{\sc Teaching and Presentation}
{\bf Teaching Assistant}       \hfill{2016.9-now}
\begin{list2}
\item Serving as teaching assistant for Stats 67 in Spring 2016, and Stats 7 in Fall 2017.
\end{list2}


{\bf Bayesian Causality} \hfill{2017.12}
\begin{list2}
\item Proposed a bayesian framework for causal relation computation. 
\item Present a poster in the workshop: From 'What If?' To 'What Next?' : Causal Inference and Machine Learning for Intelligent Decision Making, NIPS 2017
\end{list2}


%\item With Prof. Chang Liu, at Sichuan University
% %\item Stock portfolios were estabished in SSE by sorting them into quintiles according to influential factors such as EPS and money inflow rate.
% \item Historical data within 3000 trading days was crawled from web using python to backtest trading strategies. Developed and implemented several algorithms to select profitable factors as trading strategies. 
% \item Improved the automatic trading system by adding new strategies.
% \end{list2}

% {\bf EGARCH Model with Leverage Effect} \hfill{2015.10-2016.4}
% \begin{list2}
% \item With Dr. Liang Wu from Brown University.
% \item This paper find empirical evidence of Anti-leverage effect in Chinese stock market.  An analytical result regarding approximation of option prices is solved. Algorithms based on the result is developed and implemented.  
% \item Implied volatility is calculated using the option prices solved by the theorem in this paper, where it showes anti-leverage effect.
% \end{list2}



% {\bf Optimal Search Model Updated by Bayesian Method} \hfill {2015.11}


% \begin{list2}
% \item This paper aims at searching for crashed planes in ocean area. Major work include: %incorporation of environmental data,
%  simulation of crash trajectory and optimal search method using Dinkelbach's algorithm. %The probability density map of debris location is updated by Bayesian method.
% \item The result shows that the systematic plan developed using various types of searching planes helps to find the flight AF477 within 2 hours.
% \end{list2}


%\end{list2}
%{\bf Statistical Analysis on EEG and fMRI data} \hfill{in Progress}
%%%%%%
% \section{\sc Contributed Talks}

% {\bf  Seminars on Markov Chain Monte Carlo Methods} \hfill{ 2015.9-10}

% \begin{list2} 
% \item Department of Mathematics, Sichuan University
% \item Topics covered: Generation of random variable; The Metropolis Hastings Algorithm; The Two Stage Gibbs Sampler.
% \end{list2}
\section{\sc Hobbies} I am a competitve Go player (4th dan).


% \section{\sc Awards and scholarships}
% \textbullet \hspace{0.05in} Outstanding Undergraduate Student Award, Sichuan University (2016)\\ 
% \textbullet \hspace{0.05in} Honorable Mention, Mathematical Contest in Modeling (MCM), COMAP, U.S. (2015) \\
% \textbullet \hspace{0.05in} Scholarship of ‘Temasek Foundation, National University of Singapore (2014)\\
% \textbullet \hspace{0.05in}  Fellowship of Chinese National Pilot Project for Fostering Top-notch Students in Fundamental Science, Sichuan University (2013-2016) \\
% \textbullet \hspace{0.05in} First Prize Award of Mathematical Modeling Competition, Sichuan University (2013)\\



\end{resume}
\end{document}




