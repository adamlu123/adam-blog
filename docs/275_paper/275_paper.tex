\documentclass{beamer}
 
\usepackage[utf8]{inputenc}
\usepackage{array} % needed for \arraybackslash
\usepackage{graphicx}
\usepackage{adjustbox}
 \usetheme{Goettingen}
%Information to be included in the title page:
\title{Common Statistical Misinterpretations}
\subtitle{Part I}

\author{Yadong Lu}
\institute{Univeristy of California, Irvine}
% \date{Nov 19, 2017}
\date{\today}
 
 
\begin{document}
 


\frame{\titlepage}
 
\begin{frame}
\frametitle{Table of Contents}
\tableofcontents
\end{frame}

\section{Author}
\begin{frame}
\frametitle{Author:}
% \tableofcontents[currentsection]
\begin{tabular}{p{.4\textwidth} p{.6\textwidth}}
\adjincludegraphics[width=.8\linewidth,valign=t]{pic1.jpg}
&
\raggedright\arraybackslash{Sander Greenland (born January 16, 1951) is an American statistician and epidemiologist known for his contributions to epidemiologic methods, meta-analysis, Bayesian inference and causal inference, among other topics. }
\end{tabular}
Educated at the University of California, Berkeley and the University of California, Los Angeles (UCLA), he has held a professorship in epidemiology at UCLA School of Public Health since 1989, and additionally a professorship in statistics at UCLA College of Letters and Science since 1999.
\end{frame}


\section{Motivation and Methods}


\begin{frame}
\frametitle{Motivation and Methods:}

\begin{block}{Motivation}
\begin{itemize}
 \item Epidemic of shortcut definitions and interpretations that are simple, but wrong, disatrously.
 \item Misinterpretation and abuse: statistical tests, confidence intervals, statistical power. 
 \item Some journal bans all statistical tests and confidence intervals
\end{itemize}
\end{block}
\pause

\begin{block}{Methods}
\begin{itemize}
 \item<1-> Explain the meaning of significance tests, statistical power in a general way. 
 \item<2-> Review 25 common misconceptions 
\end{itemize}
\end{block}
\end{frame}


\section{Understanding Statistical Model, Hypothesis and Tests}
\begin{frame}
\frametitle{Understanding Statistical Model, Hypothesis and Tests}

\begin{block}{Statistical Model}
\begin{itemize}
 \item Mathematical representation of data variability 
 \item A set of assumptions is embodied in a model. 
\end{itemize}
\end{block}
\pause
\begin{block}{Problems:}
\begin{itemize}
 \item Models incorporate unrealistic or unjustified assumptions. 
 \item Hard to define scope of the model. (overfitting)
 \item Model in a compressed form: lead to unremarked assumptions (Think about poisson model)
\end{itemize}
\end{block}

\end{frame}


\begin{frame}
\frametitle{Understanding Statistical Model, Hypothesis and Tests}
\begin{block}{Statistical Hypothesis Tests}
\begin{itemize}
 \item Determine whether a particular effect has a specific size.
 \item Null hypothesis: target effect size has "null" value or zero effect. 
 \item Can also test whether the effect size fall in a specific range. 
 \item Focusing \textit{only} on null hypothesis test: lead to misunderstanding. Remeber we can test all kinds of target hypothesis rather just the "null" hypothesis. 
\end{itemize}
\end{block}

\begin{block}{}

\end{block}

\end{frame}

\section{Uncertainty, probability, and statistical significance}
\begin{frame}
\frametitle{Uncertainty, probability, and statistical significance}

\begin{block}{Refined Goal of Statistical Analysis}
Evaluation of certainty or uncertainty regarding the size of an effect. 
\end{block}
\pause

\begin{block}{Statistics vs. ML}
\begin{itemize}
\item Machine learning care less about the evaluation of the effect. 
\item Goal: use the effect (association) to produce prediction(MLP), or generation(Generative Adversarial Net).
\end{itemize}
\end{block} 
\end{frame}



\begin{frame}
\frametitle{Uncertainty, probability, and statistical significance}

\begin{block}{Probability: in Frequentist World}
\begin{itemize}
\item In frequentist: "Probability" refers not to hypothesis, but to quantities that are hypothetical frequencies of data patterns under an assumed statistical model
\item Think about "likelihood" : refer to the probability of the data given parameters, \textbf{not} the probability of the parameter taking on a given value.
\end{itemize}
\end{block}
\end{frame}

\begin{frame}
\frametitle{Uncertainty, probability, and statistical significance}

\begin{block}{P Value}
\begin{itemize}
\item The probability, under the null hypothesis, of obtaining a result equal to or more extreme than what was actually observed. 
 \item It is a hypothetical frequency, also known as the "observed significance level" for the test hypothesis
\end{itemize}
\end{block}
\pause
\begin{block}{Traditional Definition:}
Focusing on null hypothesis, treating all other assumptions to be correct. 
\end{block}
\pause
\begin{block}{A More General View}
P value is  a statistical summary of the compatibility
between the observed data and what we would predict or
expect to see if we knew the entire statistical model (all the
assumptions used to compute the P value) were correct.
\end{block}
\end{frame}


\begin{frame}
\frametitle{Uncertainty, probability, and statistical significance}


\begin{block}{Interpretation:}
\begin{itemize}

 \item<1-> It is true that the smaller the P value, the more unusual
the data would be if \textit{every} single assumption were correct;
but a very small P value \textit{does not} tell us which assumption
is incorrect. 
\item<2-> For example, the P value may be very small
because the targeted hypothesis is false; but it may instead
(or in addition) be very small because the some assumptions
were violated

 \item<3-> Conversely, a large P value indicates
only that the data are not unusual under the model,
but does not imply that the model or any aspect of it (such
as the targeted hypothesis) is correct;
\end{itemize}
\end{block}

\end{frame}


\begin{frame}
\frametitle{Uncertainty, probability, and statistical significance}

\begin{block}{Interpretation(contd):}
\begin{itemize}
\item The general definition of a P value may help one to
understand why statistical tests tell us much less than what
many think they do. 
\item Need assurance of every other assumptions in the model is correct - an assurance that is
lacking in far too many studies.
\end{itemize}
\end{block}

\pause

\begin{block}{P Value  Significance Level:}
\begin{itemize}
\item Significance level ($\alpha$ level): often used to refer to the "cut-off", fixed, part of study design.
\item P value: a number computed from the data. 
\end{itemize}
\end{block}

\end{frame}


% \begin{frame}
% \frametitle{Common misinterpretations of single P values}
% \begin{block}{The P value is the probability that the test
% hypothesis is true; for example, if a test of the null
% hypothesis gave P = 0.01, the null hypothesis has
% only a $1 \%$ chance of being true; if instead it gave
% P = 0.40, the null hypothesis has a $40 \%$ chance of
% being true}
% The P value assumes the test
% hypothesis is true—it is not a hypothesis probability
% and may be far from any reasonable probability for the
% test hypothesis.
% \end{block}


% \end{frame}


% \begin{frame}
% \frametitle{Common misinterpretations of single P values}
% \begin{block}{The P value for the null hypothesis is the probability
% that chance alone produced the observed association;
% for example, if the P value for the null
% hypothesis is 0.08, there is an 8 % probability that
% chance alone produced the association}

% \end{block}
% \end{frame}
\section{Moving from Tests to Estimations}

\begin{frame}
\frametitle{Moving from Tests to Estimations}
\begin{itemize}
\item we can vary the test hypothesis (or effect size) to see how P value changes. (eg: we can test $H_{0}: \mu = 0,0.5, 1$ etc) \pause
\item Effect sizes whose test produced $P > 0.05$ will define of range of sizes that would be considered more compatible with data than sizes outside the range. (this determines a range for $\mu$) 
\item The range corresponds to $95\%$ confidence interval. \pause
\item Confidence interval can be thus be seen as a convenient way of summarizing the results of hypothesis tests for many effect sizes. 
\end{itemize}
\end{frame}


\begin{frame}
% \frametitle{}
\begin{center}
\Huge Thank you!

\[
P(D_{2}|p,q) =\sum_{k_{1} = 0}^{30} \sum_{k_{2} = 0}^{30-k_{1}}\binom{100}{k_{1}}[q(1-p)]^{k_{1}}\binom{100-k_{1}}{k_{2}}[pq]^{k_{2}}\binom{100-k_{1}-k_{2}}{30-k_{1}-k_{2}}[p(1-q)]^{30-k_{1}-k_{2}}[(1-p)(1-q)]^{70}
\]
\end{center}

\end{frame}


% \begin{frame}
% \frametitle{}

% \end{frame}


% \begin{frame}
% \frametitle{}

% \end{frame}



 
\end{document}